\documentclass{article}
\usepackage[utf8]{inputenc}
\usepackage[spanish]{babel}
\usepackage{graphicx}
\usepackage[utf8]{inputenc}
\usepackage{wallpaper}
\usepackage[pdfstartview=FitH]{hyperref}
\usepackage{tabularx}
\usepackage{glossaries}
\usepackage{geometry}
 \geometry{
 a4paper,
 total={170mm,257mm},
 left=30mm,
 top=20mm,
 right=20mm,
 }
\usepackage{appendix}
\usepackage{pdfpages}
\usepackage[
backend=biber,
style=alphabetic,
citestyle=ieee
]{biblatex}
\addbibresource{references.bib} 
\begin{document}
%%%%%%%%%%%%%%%%%%%%%%%%%%%%
%%%%%      PORTADA      %%%%%
%%%%%%%%%%%%%%%%%%%%%%%%%%%%%

\begin{titlepage}


    \ThisLRCornerWallPaper{1}{imgs/fondo_tt.png} % Fondo de portada 
        \begin{center}
            \huge \textbf{Instituto Politécnico Nacional}\\*[0.3cm]%Tamano 20
            \LARGE {Escuela Superior de Cómputo}\\%Tamano 16
            \vspace{1cm}
            \begin{center}
                \LARGE{ESCOM}
            \end{center}
            %\rule{12cm}{0.5mm}\\*[0.3cm]% Línea {Longitud}{Grosor}
            \vspace{1cm}
            \large{\textit{Trabajo Terminal}}\\%Tamano 14
            \Large {\bf 'Herramienta lúdica móvil para la resolución de \\ problemas de Cálculo Diferencial'}\\*[0.4cm]
            \large {2020-A082 }\\*[0.5cm]
        \end{center}

    %\centering %Todo centrado
    %\vspace{1cm} %Espacio vertical
    \begin{center}
    %%%%  TITULO Y NÚMERO DE TRABAJO   %%%%
    %\LARGE \textbf{ Nombre del Trabajo Terminal}
    %\LARGE {\\ Número de Trabajo Terminal}
    %\vspace{1cm} %Espacio vertical
    %\LARGE \textit{Que para cumplir con la opción de titulación curricular en la carrera de:}
    %\LARGE \textbf{\\ Ingeniería en Sistemas Computacionales}
    \vspace{1cm} %Espacio vertical
    %%%%   ALUMNOS   %%%%
   \textit{Presentan:}\\ 
   \bf{Mothelet Delgado Izaird  Alexander\\
    Pineda Vieyra Itzcoatl Rodrigo\\}
    
    \vspace{1cm} %Espacio vertical
    \textit{Directores:}\\ 
    \bf{Lorena Chavarría Baez\\
    Ruiz Ledesma Elena Fabiola \\}
    \end{center}
    \null\vfill\hfill 6 de Junio de 2021
\end{titlepage}

\tableofcontents
\section{Introducción}
\subsection{Motivación}
Estadísticas del INEGI sobre deserción escolar, muestran niveles realmente impactantes en la Ciudad de México, conforme se avanza en el nivel educativo también se ve un aumento en la tasa de deserción académica. Esta problemática es multifactorial, debido a que hay muchos factores diferentes por los cuales los estudiantes abandonan sus estudios. Dentro del ámbito académico hay factores como no contar con los conocimientos previos necesarios, los métodos de enseñanza y la falta de motivación, por mencionar algunos. 

La motivación no solo es un factor de la deserción escolar, sino que está involucrada en el proceso de enseñanza-aprendizaje. El trabajo en torno a la motivación es muy amplio y con muchas aristas dentro del ámbito educativo. Una de estas aristas es el aprendizaje del estudiante. En relación a éste, diversos estudios, muestran que la motivación está involucrada en el proceso de enseñanza-aprendizaje \cite{MISBAH201579} \cite{2Mot}. Para obtener aprendizajes significativos los alumnos deben sentir cierta motivación hacia el estudio. La cuestión es ¿Cómo motivar a los alumnos para obtener los aprendizajes deseados? Según la teoría de la Autodeterminación (SDT por sus siglas en inglés) la motivación intrínseca (actuar porque el acto en sí es disfrutable, sin la necesidad de ningún estímulo externo) es la que genera mejores aprendizajes. Esto se puede dar presentando actividades y desafíos nuevos o interesantes. Sin embargo, no siempre se puede depender de la motivación intrínseca, cada estudiante es diferente y no encontrarán estimulantes las mismas actividades. Es entonces que se debe recurrir a la motivación extrínseca, lo que incluye actuar por recompensas. SDT indica que entre más se internalice y se entienda el por qué realizar cierta actividad y su utilidad, más autodeterminado se vuelve nuestro actuar, no es igual de bueno que estar motivado intrínsecamente, pero es la segunda mejor opción. 

Con el presente trabajo terminal se pretende apoyar al estudiante en la motivación hacia la resolución de ejercicios en la disciplina de matemáticas, en particular en aritmética, algunos aspectos de álgebra y cálculo diferencial al proporcionar una aplicación móvil con elementos de gamificación para motivar a los estudiantes.   
Varios autores sostienen que la incorporación de la tecnología está marcada por la motivación que el sujeto o esa tecnología generen [3], [4], [5], [6]. En esta dirección distintos autores [7], [8] sostienen que existe relación positiva entre la motivación generada por los estudiantes frente a una estrategia propuesta por el profesor y los resultados de su aprendizaje. Por ello, se considera de vital importancia el uso de estrategias pedagógicas que motiven en el alumno el interés por su aprendizaje [9], [10]. Sin embargo, esta relación entre el factor motivacional por el uso de la tecnología y los demás resultados de los procesos de la enseñanza y el aprendizaje no siempre es clara dado que éstos, son procesos complejos que involucran diversos aspectos tanto del que aprende y del que enseña como del contexto en el que se dan estos procesos [11], [12]. Es por lo anterior que el solo uso de la tecnología no es garantía para que el estudiante se encuentre motivado para usarla, de ahí el que se requiera de otro ingrediente que en el presente reporte de TT se trabaja y es la gamificación.


\subsection{Planteamiento del Problema}
En el ámbito de la educación, es todo un reto el crear e implementar métodos de enseñanza que motiven a los estudiantes a realizar sus tareas. Dicha problemática se da en todos los niveles educativos. Los alumnos difícilmente encuentran utilidad a lo que hacen más allá de aprobar el curso. Dada la situación actual de confinamiento, debido a la pandemia a causa del COVID-19, los estudiantes de todos los niveles necesitan contar con un actuar autodidacta, el cual se puede facilitar con aplicaciones móviles, que permitan desarrollar competencias matemáticas. Al hacerlo por medio de juegos aumenta la motivación de los estudiantes.
Las aplicaciones móviles actuales como: Ahora si paso, Prepa en un 2x3 Examen, no suelen incorporar aspectos lúdicos, o que promuevan la asimilación de los contenidos vistos, siendo esto un aspecto necesario para temas que no despiertan interés en el alumno. Dado que, en el ámbito de las aplicaciones móviles educativas para nivel medio superior en adelante, se enfocan más en generar ganancias al ofrecer versiones de prueba gratuitas y bloqueando el resto de preguntas a menos que se pague una suscripción o por la aplicación completa. 
Si bien existen plataformas educativas bastante competentes como Khan Academy y Brilliant.org. Esta última desafortunadamente, no está en español y tiene un costo mensual, mientras que Khan academy no incorpora aspectos de gamificación. 
Los alumnos de media superior en adelante cuentan con opciones muy limitadas en cuanto a cómo aprender matemáticas de ese nivel. El método más empleado es el de resolver varias listas de ejercicios, empezando desde lo más elemental hacia lo más complejo. Este método, aunque efectivo, tiene la problemática de que los alumnos no disfruten de realizar dichos ejercicios o no vean la utilidad de realizarlos. 
Es por eso que se propone como apoyo a dicho método una aplicación móvil, que se apoye en la gamificación para la realización de los ejercicios. Basados en la teoría de la Autodeterminación, se diseñó un modelo de motivación para guiar la aplicación,
Los usuarios podrán tener diferentes motivos para descargar la app, pero descargarla por cuenta propia ya implica cierta motivación y objetivo. La aplicación, así como muchos juegos, contará con desafíos y recompensas (títulos, iconos, puntuaciones), con la finalidad de atraer y retener a los usuarios, satisfaciendo el nivel más bajo de motivación extrínseca de la teoría. La aplicación al contar con un generador de preguntas podrá proveer una gran variedad de problemas, manteniendo el desafío. Finalmente se pretende demostrar la utilidad de los problemas resueltos al aplicar ese conocimiento en problemas de optimización, con esto se pretende que los usuarios descubran la utilidad de una parte que corresponde al  Cálculo diferencial. 

\subsection{Objetivos}
\subsubsection{Objetivo General}
Desarrollar  una aplicación de software, basada en el enfoque de gamificación, para motivar al estudiante de nivel medio superior y superior en la resolución de ejercicios aritméticos y algebraicos requeridos para trabajar problemas de  optimización.

\subsubsection{Objetivos Específicos}

\begin{enumerate}
    \item Crear un módulo generador de ejercicios(Aritméticos, algebraicos, diferenciales, problemas de optimización), junto con un evaluador de expresiones.
    \item Fomentar el desarrollo del cálculo mental de los usuarios.
    \item Diseñar un modelo de gamificación específico a la aplicación basado en la teoría de autodeterminación. \item Incorporar elementos de juegos para mejorar la experiencia del usuario.
\end{enumerate}
\subsection{Estado del arte}
Se han realizado numerosos estudios sobre gamificación en la educación [13] [14], su impacto en la motivación de los estudiantes es alentador, independientemente de factores demográficos como edad o género [15]. 
Ciertos elementos han  mostrado consistentemente un efecto positivo en la motivación de los estudiantes [16], [17], particularmente en aquellos que ya están familiarizados con videojuegos [15]. Cabe destacar, que la mayoría de los resultados revisados son empíricos, con poblaciones pequeñas (entre 100 y 200 estudiantes) y utilizando distintos métodos de investigación, independientemente todos los estudios revisados reportaron un incremento en la motivación e interés de los estudiantes.
La gran mayoría de estudios de gamificación, particularmente en matemáticas, se han realizado a nivel de primaria [18], [19], [20] . Sin embargo, la gamificación ha mostrado resultados alentadores aún en niveles superiores [21], [22]. Particularmente se ha destacado la utilidad de la gamificación en ambientes de aprendizaje virtuales y educación en línea [23], particularmente en el caso de México [24].
En el caso de la materia Cálculo en nivel superior, un estudio [25], destaca la utilidad de la gamificación como estrategia para aumentar la motivación y emoción que promueve el aprendizaje. Concluye también que su potencial educacional puede ser reforzado al incluir procesos de evaluación del desempeño y metaevaluación. 
Otro antecedente en México [26], donde se realizaron actividades gamificadas por equipos con plastilina y foami, seguida de una actividad integradora para demostrar lo aprendido. Se implementó una tabla de avance donde se registraba el progreso de los equipos.
El potencial de la gamificación en la educación es enorme, los cambios en la educación suelen ser muy lentos. La educación debe buscar cómo adaptarse a la modernidad tecnológica, al fácil y rápido acceso a información. La gamificación busca explotar las nuevas tecnologías para dar experiencias de aprendizaje que se adapten mejor a las necesidades y características de los estudiantes de hoy en día. 
\subsection{Propuesta de solución }
\section{Marco Teórico}
\subsection{Motivación}
\subsection{Gamificación}
\section{Análisis}
\subsection{Requisitos}
\subsubsection{Requisitos funcionales} 
\subsubsection{Requisitos no funcionales} 
\subsubsection{Requisitos de usuario} 
\subsubsection{Requisitos del sistema}
\subsubsection{Requisitos de dominio}
\subsection{Análisis de interfaces}
\subsection{Narrativa de la Base de Datos}

\section{Diseño}
\subsection{Diseño de Base de Datos}
\subsection{Casos de Uso}
\subsection{Diseño de Interfaces}
\subsection{Diseño de Problemas}
\subsection{Metodología}
\end{document}
